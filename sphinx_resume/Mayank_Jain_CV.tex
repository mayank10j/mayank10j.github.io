%%%%%%%%%%%%%%%%%%%%%%%%%%%%%%%%%%%%%%%%%
% Medium Length Professional CV
% LaTeX Template
% Version 2.0 (8/5/13)
%
% This template has been downloaded from:
% http://www.LaTeXTemplates.com
%
% Original author:
% Trey Hunner (http://www.treyhunner.com/)
%
% Important note:
% This template requires the resume.cls file to be in the same directory as the
% .tex file. The resume.cls file provides the resume style used for structuring the
% document.
%
%%%%%%%%%%%%%%%%%%%%%%%%%%%%%%%%%%%%%%%%%

%----------------------------------------------------------------------------------------
%	PACKAGES AND OTHER DOCUMENT CONFIGURATIONS
%----------------------------------------------------------------------------------------

\documentclass{resume} % Use the custom resume.cls style

\usepackage[left=0.75in,top=0.6in,right=0.75in,bottom=0.6in]{geometry} % Document margins

\name{Mayank Jain} % Your name
\address{AI Architect \\ Author \\ Tinkerer } % Your name
\address{ https://mayankjain.co.in \\ https://www.linkedin.com/in/mayankjain7/  \\ https://github.com/mayank10j} % Other Contacts
\address{House no. 135 First Floor , Sector 43 , Near Sector 43 Jain Mandir, Gurgaon, Haryana 122003} % Your address
\address{(+91)~9650419555 \\ mayank10.j@gmail.com }  % Your phone number and email

\begin{document}
\begin{rSection}{Book Publication}
\begin{itemize}
\item  \emph{Ensemble Learning for AI Developers}, https://www.springer.com/de/book/9781484259399 

\end{itemize}

\end{rSection}
%----------------------------------------------------------------------------------------
%	WORK EXPERIENCE SECTION
%----------------------------------------------------------------------------------------



\begin{rSection}{Work Experience}

%------------------------------------------------

\begin{rSubsection}{Sapient Consulting Pvt. Ltd. (PUBLICIS.SAPIENT) }{Aug 2014 - Present}{Manager Technology}{KeplerLab}
\textbf{Role:} Currently working as Manager Technology in Sapient India Experiments Lab: Kepler.
Working as AI Architect delivering solutions using emerging technologies like 
Machine learning, Deep Learning NLP, Computer vision, AR, VR etc. 

\textbf{Some Projects:} 

\item \textbf{Katna}:
Developed an tool for automating common video keyframe extraction and
Image Autocrop and image resize tasks, which 
helped save thousands of man hours across multiple projects.\\
Technologies: Python, Image-processing, video-keyframe-extraction , OpenCV, FFMPEG \\
\textit{https://github.com/keplerlab/katna}

\item \textbf{Alternat}:
alternat is a collection of open-source toolsets with the ambition of
lowering the barrier of adopting accessibility solutions.
alternat helps to generate default intelligible alternative text for images in websites.
Being used for solving website accessibility. \\
Technologies involved: Python, PyTorch, Azure and Google cloud api, Nodejs,Javascript,Docker \\
\textit{https://alternat.readthedocs.io/}

\item \textbf{Idea2Life}:
Real time recognition of Web components from Templates and sketches to make 
responsive websites,Powered by Deep learning :\\
Technologies: Yolov2/Darknet, Tensorflow, Python, Nodejs,Javascript,Docker \\
\textit{https://github.com/keplerlab/idea2life}

\item \textbf{Interactive Mirror Installation (IMI)}:
Deep Learning based Smart Retail assistance for physical store. 
Solving inventory visibility problem for physical retail store.
Technologies: Yolov2/Darknet, Tensorflow/Keras,Faiss,
Object Detection, Object classification, apparel classification, 
content based image retrieval, Python, Nodejs,Javascript,Docker  \\
\textit{http://experiencesutra.com/projects/the-mirror-reimagined\char`_-a-smart-shopping-assistant-for-the-retail/}
%\textit{http://experiencesutra.com/experiments/improving-colour-based-filtering-results-for-interactive-mirror-installation/}

\item \textbf{AI in Content Creation}:
Use of ML/Deep learning for delivering efficiency to Sapient India content processes.
Technologies used: NLP, Deep learning based text summarization,
headline generation from article text, document tagging, topic modelling,
content aware image cropping etc.
Technologies: Tensorflow/Keras, Python, Nodejs,Javascript,Docker 

\item \textbf{Taj AR App}:
Combining Deep learning with Augmented reality for interactive tour guide using iOS coreML and ARKit \\ 
Technologies: iOS CoreML and ARKit, Tensorflow,Darknet, Python, Swift \\
\textit{http://experiencesutra.com/experiments/reimagining-the-tour-guide/}


\item \textbf{Project CHAMELEON}:
Using Deep neural networks using Torch for transfer of custom styles to an web app\\ 
Technologies: Torch,Lua,Nodejs, Style transfer, Javascript\\
\textit{http://experiencesutra.com/experiments/project-chameleon/}

\item \textbf{Neural Creativity, Hand drawn sketch Recognition}:
Use of Deep convolutional neural Network to recognizes hand drawn sketches
with more than 70 percent accuracy. Use of custom html5 game engine to bring
your childhood imagination alive\\ 
Technologies: Caffe,Python,OpenCV,C++,Nodejs,Javascript\\
\textit{https://github.com/keplerlab/neuralCreativityServer} \\
\textit{https://www.youtube.com/watch?v=VajzcTbMobA}

\item \textbf{Autolysis, Combining Art with technology}:
Collaboration project with renowned artist Asim waqif to use message of decay
using technology. Installation uses CV techniques to bring about his ideas to life.\\
Technologies: C++,OpenCV,Javascript\\
\textit{https://www.youtube.com/watch?v=8NHU-L58fdw}



\item \textbf{ObjecTable, augmented everyday object experience}:
One of the very first experience projects that were built out of Kepler Lab,
Table Recognizes what is on top of it using simple webcam and changes it UI accordingly.\\
Technologies: OpenCV,C++,Nodejs,Javascript\\ 
\textit{http://awards.designforexperience.com/gallery/2014/experience-that-makes-a-difference/sapientnitro-0}


\end{rSubsection}

\begin{rSubsection}{Samsung Research Institute -Delhi (SISC) }{Feb 2013 - Aug 2014}{Senior Software Engineer}{Advanced Research Group}
\item Point Cloud Processing Research Project (Aug 2013 - Aug 2014): 
Research project in Collaboration with IIT Delhi (Prof. Prem K Kalra ,Dr. Subodh Kumar and Dr. Subhashish Banerjee) \\
Technologies: PCL, Eigen, OpenGL, OpenCV, C++
\item Visual Attention Modelling System (March 2013 - Aug 2013): 
Detection of Persons attention while in front of Display using combination of face and eye tracking\\
Technologies: OpenCV, C++

\end{rSubsection}


\begin{rSubsection}{TCS (TCS Innovation Labs, Delhi)}{December 2009 - February 2013}{Researcher}{Computer vision and Robotics Group}
\item Topological SLAM with visual Finger printing (Aug 2012 to Feb 2013):  
 Building visual signature of places already seen to build topological SLAM, Use of Bag of words approach. Tools: OpenCV , C++ , ROS 
\item Face based Video Indexing (Early 2012): 
Building video indexing system based on face tracking. Tools: OpenCV , C++ , Qt
\item Real Time Human Avatar Model Tracking System (2011): 
A Kinect, mounted in front of a body, was used to iteratively search and track the human body in each video frame. Image and depth data were manipulated and transformed to local co-ordinates of avatar model and used in real time tracking of human body in virtual world. Tools: OpenNI , OpenCV , Visual C++
\item Browsing Behaviour Analysis in Retail Stores (Mid 2011): 
A Kinect mounted in a retail store was used to analyze the browsing behaviour of customers which includes Pick, Drop, Purchase of goods kept on shelves. Depth segmentation of 3D environment together with unsupervised clustering of human poses applied to AI core engine for classification of different activities.  Tools: OpenCV , QT , C++
\item Hand Raise and Face Motion detector (2010): 
Used as module in one of Virtual Reality product for detecting hand raise and face and shoulder motion detection of a person in front of webcam.  Tools: OpenCV, QT  

\end{rSubsection}

\end{rSection} 

\begin{rSection}{Speaker Roles}

    \begin{itemize}
    \item GIDS 2019: Primer to Ensemble Learning\\s
    \textit{https://www.developermarch.com/developersummit/session.html?insert=AlokKumar1}
    \item Frequent speaker at Sapient Sapestart, new joiners induction programme at Publicis Sapient.
    \item Speaker and Organizer for Multiple webinars and Sessions for Deep learning, Machine learning inside Publicis Sapient.
    \item GIDS 2016: Speaker in Asia\textsc{\char13}s Largest Developer conference \emph{Great India Developer Summit 2016} on topic \textit{Machine Learning for Everyday experiences.} \\ 
    \textit{https://web.archive.org/web/20160419172630/http://www.developermarch.com\\/developersummit/session.html?insert=MayankJain1}
    \end{itemize}
\end{rSection}



%\pagebreak[4]

\begin{rSection}{Patents}

\begin{itemize}
\item \textit{Method and Apparatus for Environmental profile generation at Samsung Research Delhi };\\ http://www.google.com/patents/WO2016022008A1
\item \textit{A System and Method for Estimating Human Upper Body Pose from Single Image};  Application No. 1831/MUM/2011
\item \textit{A System and Method for Tracking Multiple Faces with Appearance Modes and Reasoning Process}; Application No. 1959/MUM/2011
\item \textit{A System and Method for Face based Video Indexing in a Video}; Application No. 2254/MUM/2011
\end{itemize}


\end{rSection}

\begin{rSection}{Publications and Talks}

\begin{itemize}


\item Nipun Pande, Mayank Jain, Dhawal Kapil and Prithwijit Guha,\emph{The Video Face Book}, The 18th International Conference on Multi Media Modelling (MMM 2012), Klagenfurt (Austria), January 4-6, 2012 
\item Prithwijit Guha, Mayank Jain, Nipun Pande and Tavleen Oberoi, \emph{Multiple Face Tracking with Appearance Modes and Reasoning}, The 15th International Conference on Image Processing, Computer Vision and Pattern Recognition (IPCV 2011), pp. 375-380, Las Vegas, July 18-21, 2011
\item \emph{Using blockchain technology for loyalty schemes} \\ 
\textit{https://www.thehindubusinessline.com/catalyst/using-blockchain-technology-for-loyalty-schemes/article23700167.ece}

\end{itemize}

\end{rSection}

\begin{rSection}{Internships}

    %\begin{itemize}
    
    \begin{rSubsection}{National ICT Australia , IIIS Griffith University }{May 2008 - July 2008}{Visiting Researcher}{Smart Applications for Emergencies (SAFE) Group}
    \item Development of Heuristic for solving Boolean satisfiability problem solver using stochastic local search approaches. 
    
    \end{rSubsection}
    
    \begin{rSubsection}{Arcelor Mittal Steel plant,Kazakhstan}{May 2007 - June 2007}{Intern}{ ArcelorMittal Temirtau}
    \item Hands on experience By Network Configuration and Monitoring of Temirtau City network.
    
    \end{rSubsection}
    
    \end{rSection}
    
%----------------------------------------------------------------------------------------
%	TECHNICAL STRENGTHS SECTION
%----------------------------------------------------------------------------------------

\begin{rSection}{Technical Strengths}

\begin{tabular}{ @{} >{\bfseries}l @{\hspace{6ex}} l }
Computer Languages & C, C++ , Python , Javascript\\
Skills & Computer vision, Natural Language processing, Machine Learning/Deep Learning \\
Platforms & GNU/Linux, Windows Environments , Mac OS X\\
Tools & PyTorch, Tensorflow, Keras, Docker, OpenCV, Google Cloud, AWS, Git,\\ & NumPy/Scipy/Pandas/matplotlib, ROS, MS Visual Studio, Xcode, \LaTeX
\end{tabular}

\end{rSection}

%----------------------------------------------------------------------------------------
%	EDUCATION SECTION
%----------------------------------------------------------------------------------------

\begin{rSection}{Education}


{\bf Laxmi Niwas Mittal Institute of Information technology, Jaipur} \hfill {\em 2005-2009} \\ 
B.Tech. in Computer Science \& Engineering \\
Overall CGPA: 7.73


{\bf Kendriya Vidyalaya Number 1 Jaipur, CBSE Board } \hfill {\em 2003-2004} \\ 
AISSCE 2004 (XII) in Science Maths   \\
Overall Percentage: 79\% 

{\bf Kendriya Vidyalaya Number 1 Jaipur, CBSE Board } \hfill {\em 2001-2002} \\ 
AISSE 2002 (X)    \\
Overall Percentage: 70.2\% 


%XII (79.0%) from Kendriya Vidyalaya Number 1 Jaipur, CBSE Board
%X (70.2%) from Kendriya Vidyalaya Number 1 Jaipur CBSE Board



%----------------------------------------------------------------------------------------
%	Personal Details
%----------------------------------------------------------------------------------------

\begin{rSection}{Personal Details}

\begin{tabular}{ @{} >{\mdseries}l @{\hspace{6ex}} l }
Date of Birth & 12th May 1987 \\
Sex & Male \\
Nationality & Indian \\
Marital status & Married \\
Languages Known & English , Hindi \\ 
Interests & Reading ,Writing , Programming , Current affairs, Economics%, Psychology
\end{tabular}


\end{rSection}




\end{rSection}
%----------------------------------------------------------------------------------------
%	EXAMPLE SECTION
%----------------------------------------------------------------------------------------

%\begin{rSection}{Section Name}

%Section content\ldots

%\end{rSection}

%----------------------------------------------------------------------------------------

\end{document}
